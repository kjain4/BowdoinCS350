\documentclass[11pt,twocolumn]{article} 
\usepackage{fullpage}


\begin{document}

\title{Calculating Flow on Grids}
\author{William Richard}
\maketitle 


\abstract{This writeup summarizes the design and findings of my first
  project in csci350, Spring09. }

\section{Overview}

This project acts as a miniature GIS and lets the user execute the
followig commands: 
\begin{verbatim}
gis>help
render grid.asc
r grid.asc
flowdir elev.asc dir.asc
fd elev.asc dir.asc
flowaccu elev.asc dir.asc accu.asc
fa elev.asc dir.asc accu.asc
quit
gis>
\end{verbatim}

The \texttt{flowdir} or \texttt{fd} program computes the flow direction grid
corresponding to an elevation file. 


The \texttt{flowaccu} or \texttt{fa} program computes the flow accumulation grid
corresponding to an elevation file and a flow direction file.


\section{Algorithm and analysis}

\paragraph{Flow direction. }
The flow direction computation works by
iterating in order over the grid, and computing, for each point
$(i,j)$, the neighbor of minimum height. A neighbor is considered a
point $(i+/-1, j+/-1)$.

{\bf Space:} My algorithm is very space-aware. It stores 2 grids of
shorts, one for elevation and one for the flow direction, so the space
usage will be 2*sizeof(short)*n. 

{\bf Time: } The running time should be on $O(n)$.


\paragraph{Flow accumulation.} The flow accumlaion algorithm works by
sorting the data it takes in by elevation, decending, so that the
first index holds the point of highest elevation.  It then goes
through the array, from highest to lowest elevation, passing the
accumulated flow in the flow direction.  Since we are going in
decending elevation, it is guaranteed that a given point has
accumulated all the flow it will accumulate before it passes that flow
downstream.  It then resorts the array in r,c order, so that it can be
written in order.

{\bf Space:} Each point needs to hold its row, column, elevation,
accumulation so far, and a pointer to its downhill neighbor.
Therefore, the space usage is relatively high.  It is $4*sizeof(short)
+ sizeof(Point*)$.

{\bf Time: } The running time is $O(n \log n)$ for each sort, and the
passing of flow runs in $O(n)$ time, so the total running time is
$O(3n /2logn)$.


\section{Experimental evaluation}

To determine the performance in practice of my algorithms I performed
a careful experimental evaluation.

{\bf Platform.} The code was written in C, compiled with \texttt{gcc
  4.0.2, -O3 -DNDEBUG}, 64 bits enable, and executed on a Mac Quad
Core 2 x 2.8 GHz Intel Xeon processor, 4GB RAM running Leopard.

{\bf Datasets: } The test datasets are real-life grid terrains ranging
from 1 million to 1,000 million points.  They are given in
Table~\ref{tbl:datasets} below.


\begin{table}[htp]
    \centering{
    \begin{tabular}{|l|r|}
      \hline 
      Dataset & nb. of points \\
      \hline 
      \hline 
      {Kaweah}        &      $1.6 \cdot 10^6$ \\
      {Sierra}        &      $9.5 \cdot 10^6$ \\
      {East-Coast USA (usadem2)} & $ 246 \cdot 10^6$ \\
      {Washington State}    & $1,066 \cdot 10^6$ \\
      \hline
    \end{tabular}
  }
  \caption{Size of terrain datasets.}
  \label{tbl:datasets}
\end{table}
      

The running times for \texttt{flowaccu} and \texttt{flowdir} are
summarized in Tables~\ref{tbl:rundir} and \ref{tbl:runaccu}. Don't
worry about the placement of tables cause Latex places the figures and
table the best it can. That's not always where you want them.

\begin{table}[htp]
    \centering{
    \begin{tabular}{|l|r|r|}
      \hline 
      Dataset & time & CPU\\
      \hline 
      \hline 
      {Kaweah}        &     1.17 & 99.2 \\
      {Sierra}        &      8.30 & 99.2 \\
      {East-Coast USA (usadem2)} & 237.49 & 98.6 \\
      {Washington State}    & ? & ? \\
      \hline
    \end{tabular}
  }
  \caption{Running times (seconds) and CPU-utilization for
    \texttt{flowdir}.}
  \label{tbl:rundir}
\end{table}


\begin{table}[htp]
    \centering{
    \begin{tabular}{|l|r|r|}
      \hline 
      Dataset & time & CPU\\
      \hline 
      \hline 
      {Kaweah}        &     1.77 & 97.5 \\
      {Sierra}        &      11.86 & 99.5 \\
      {East-Coast USA (usadem2)} & 31013.26 & 63.1 \\
      {Washington State}    & ? & ? \\
      \hline
    \end{tabular}
  }
  \caption{Running times (seconds) and CPU-utilization for
    \texttt{flowaccu}.}
  \label{tbl:runaccu}
\end{table}

%%in case you want to put a figure
% \begin{figure}[hptb]
%   \centering
%   \includegraphics [width=0.45 \linewidth] {figures/kaweah2.pdf} 
%   \includegraphics [width=0.45 \linewidth]{figures/sierra.pdf}


\section{Conclusion}
Running flow accumulation on grids is easy to impliment, but not terribly efficient.

\end{document}


